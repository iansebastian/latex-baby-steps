\documentclass{article}

\usepackage{graphicx}
\usepackage{subcaption}

\title{Displaying Images With \LaTeX}
\author{Ian Sebastian}

\begin{document}
	\maketitle
	
	Figure \ref{fig:H-R} shows a Hertzsprung-Russell diagram.
	
	\begin{figure}[h!]
		\centering
		\includegraphics[width=0.5\linewidth]{H-R.jpg}
		\caption{This is meant to be an example of a Hertzsprung-Russell diagram.}
		\label{fig:H-R}
	\end{figure}

	\newpage
	\begin{figure}[h!]
		\centering
		\begin{subfigure}[b]{0.4\linewidth}
			\includegraphics[width=\linewidth]{H-R.jpg}
			\caption{H-R Diagram One.}
		\end{subfigure}
		\begin{subfigure}[b]{0.4\linewidth}
			\includegraphics[width=\linewidth]{H-R.jpg}
			\caption{H-R Diagram Two.}
		\end{subfigure}
		\caption{Two identical H-R diagrams. But corporate still needs you to find the difference between them.}
		\label{fig:H-R_2}
	\end{figure}

	\newpage
	\begin{figure}
		\centering
		\begin{subfigure}[h]{0.2\linewidth}
			\includegraphics[width=\linewidth]{H-R.jpg}
			\caption{One H-R}
		\end{subfigure}
		\begin{subfigure}[h]{0.2\linewidth}
			\includegraphics[width=\linewidth]{H-R.jpg}
			\caption{Two H-R}
		\end{subfigure}
		\begin{subfigure}[h]{0.2\linewidth}
			\includegraphics[width=\linewidth]{H-R.jpg}
			\caption{Three H-R}
		\end{subfigure}
		\begin{subfigure}[h]{0.5\linewidth}
			\includegraphics[width=\linewidth]{H-R.jpg}
			\caption{Toby Flenderson.}
		\end{subfigure}
		\caption{Four H-R diagrams in two different sizes.}
		\label{fig:H-R_4}
	\end{figure}

	\paragraph{I'm too tired to go on, so that's it for today, folks!}
\end{document}