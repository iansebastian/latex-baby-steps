\documentclass{article}

\usepackage{amsmath}

\title{Some \LaTeX \, Math Display}
\author{Ian Sebastian}

\begin{document}

	\maketitle
	
	\section{Inline Math}
	This formula $f(x) = x^2$ is an example of inline math display. So is $L = \sigma T^4 4\pi^2$. Meh, boring. Moving on...
	
	
	\section{The Equation and Align Environment}
	
	\subsection{The Equation Environment}
	It allows you to do this:
	\begin{equation*}
		1 + 2 = 3
	\end{equation*}
	\begin{equation*}
		1 = 3 - 2
	\end{equation*}
	
	\subsection{The Align Environment}
	Is the {\tt \textbackslash equation} environment, but better. Just take a look at this:
	\begin{align*}
		1 + 2 &= 3 \\
		1 &= 3 - 2
	\end{align*}
	
	
	\section{More Math Inputting to Flex Those \LaTeX Muscles}
	\subsection{Please Bear With Me}
	\begin{align*}
		f(x) &= x^2 						 \\
		g(x) &= \frac{1}{x} 				 \\
		F(x) &= \int^{a}_{b} \frac{1}{3} x^3 \\
		G(x) &= \frac{1}{\sqrt{x}}			 \\
	\end{align*}
	
	\subsection{Matrices}
	This is an example of a {\tt Matrix} in \LaTeX.\\
	\textbf{Gentle Reminder: }This will only work within math environments described above.
	\begin{align*}	
		\left[
		\begin{matrix}
			1 & 2 \\
			2 & 1 \\
		\end{matrix}
		\right]
	\end{align*}
	
	That's it. That's a matrix. Happy?
	
\end{document}